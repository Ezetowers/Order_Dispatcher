\author{} % Lo pongo para que el warning no moleste :p
\setlength{\unitlength}{1cm} %  Especifica la unidad de trabajo
\thispagestyle{empty}

\begin{picture}(18,0)
\put(0,0){\includegraphics[width=1.5cm, height=3cm]{Imagenes/Logo1.png}}

\put(10.5,0){\includegraphics[width=3cm, height=3cm]{Imagenes/Logo2.png}}

\end{picture}
\\[1.5cm]
\begin{center}
	\textbf{{\Huge Facultad de Ingeniería \\ Universidad de Buenos Aires}}
    \\[2cm]
	{75.61 Taller de Programación III}\\[0.5cm]
	{TP N$^{\circ}$2 - Ejercicio de Colas (RabbitMQ) (Correciones)}\\[2cm]
\end{center}

\begin{flushleft}
	\textbf{Profesor: Andrés Veiga} \\
    \textbf{JTP: Pablo Roca} \\[1cm]
	\textbf{Integrantes:} \\[1cm]

	\begin{tabular}{|c|c|c|}
		\hline
		\textbf{\normalsize Padrón} & \textbf{\normalsize Nombre} 
                                    & \textbf{\normalsize Email} \\
		\hline
		\normalsize 89579 & \normalsize Torres Feyuk, Nicolás R. Ezequiel 
                          & \normalsize ezequiel.torresfeyuk@gmail.com \\
		\hline
	\end{tabular}
\end{flushleft}
\date{} % Hace que no se imprima la fecha en la cual se compilo el .tex
